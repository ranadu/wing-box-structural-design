\section{Introduction}

The wing is one of the most highly loaded structural components of a fixed-wing aircraft. During flight, it must resist significant bending moments generated by aerodynamic lift, as well as shear and torsional loads arising from maneuvering, gusts, and asymmetric conditions. The primary structure responsible for carrying these loads is the wing box.

The wing box typically consists of upper and lower load-carrying skins connected by spars and internal structure, forming a closed-section beam. This configuration provides high bending stiffness and torsional rigidity while maintaining structural efficiency. As a result, the wing box plays a central role in determining the overall strength, stiffness, and weight of the wing.

The objective of this project is to develop a conceptual structural design of an aircraft wing box suitable for early-stage aerospace design evaluation. Rather than focusing on detailed stress sizing or certification-level analysis, this study emphasizes structural layout, load-path reasoning, and design justification. A parametric CAD model is used to support the structural concept and to visually demonstrate how loads are transmitted through the structure.

This project is intended as a portfolio-quality engineering study, demonstrating the ability to reason about aircraft primary structures, make defensible design assumptions, and communicate engineering decisions clearly. The scope reflects realistic industry practice at the conceptual design level, where geometry, architecture, and structural philosophy are established prior to detailed numerical analysis.

The report is organized as follows. Section 2 defines the design context and key assumptions used throughout the study. Section 3 presents representative flight load cases used to inform the structural design. Section 4 details the wing box geometry and internal structure, supported by CAD visualizations. Manufacturing considerations, limitations, and opportunities for future refinement are discussed in subsequent sections before concluding remarks are provided.