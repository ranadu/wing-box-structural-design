\section{Wing Box Structural Design}

\subsection{Role of the Wing Box}

The wing box serves as the primary load-carrying structural element of the aircraft wing. It is responsible for resisting bending moments generated by aerodynamic lift, transferring shear loads between structural members, and providing torsional rigidity to prevent excessive wing twist.

In conceptual aircraft design, the wing box architecture has a dominant influence on structural performance. Decisions regarding spar placement, skin separation, and cross-sectional geometry largely determine bending stiffness, torsional stiffness, and overall structural efficiency. As such, the wing box is treated as the central structural focus of this study.

%------------------------------------------------

\subsection{Structural Concept}

The wing box was designed as a closed-section beam consisting of upper and lower skins connected by front and rear spar webs. This configuration is widely used in aircraft primary structures due to its ability to efficiently resist combined bending and torsional loads.

A hollow interior was incorporated to reduce structural weight while maintaining stiffness. The resulting geometry provides a favorable balance between load-carrying capability and material efficiency, consistent with early-stage aerospace structural design practice.

%------------------------------------------------

\subsection{Geometric Configuration}

The wing box geometry was developed using parametric CAD modeling. A straight, prismatic section was selected to represent a portion of the wing away from geometric discontinuities such as the wing root and control surface cutouts.

Figure~\ref{fig:wingbox_iso} presents an isometric view of the wing box, illustrating the overall configuration and closed-section geometry.

\begin{figure}[H]
    \centering
    \includegraphics[width=0.85\textwidth]{figures/wingbox_isometric.png}
    \caption{Isometric view of the wing box structure}
    \label{fig:wingbox_iso}
\end{figure}

%------------------------------------------------

\subsection{Planform and Cross-Section Views}

The planform and front views of the wing box are shown in Figures~\ref{fig:wingbox_top} and~\ref{fig:wingbox_front}, respectively. These views highlight the spanwise extent of the structure and the separation between the spar webs.

The depth of the wing box plays a critical role in resisting bending moments, as increased separation between the upper and lower skins leads to reduced bending stress for a given load. Spar placement is therefore a key design variable influencing structural performance.

\begin{figure}[H]
    \centering
    \includegraphics[width=0.85\textwidth]{figures/wingbox_top.png}
    \caption{Top view of the wing box geometry}
    \label{fig:wingbox_top}
\end{figure}

\begin{figure}[H]
    \centering
    \includegraphics[width=0.7\textwidth]{figures/wingbox_front.png}
    \caption{Front view of the wing box showing spar spacing and box depth}
    \label{fig:wingbox_front}
\end{figure}

%------------------------------------------------

\subsection{Internal Structure and Section Analysis}

A section analysis was performed within the CAD environment to visualize the internal structure of the wing box. The resulting cross-sectional view is shown in Figure~\ref{fig:wingbox_section}.

This view clearly illustrates the hollow interior of the wing box and the load-carrying skins and spar webs. The closed-section geometry provides significant torsional rigidity, which is essential for maintaining aerodynamic performance and preventing excessive wing twist under load.

\begin{figure}[H]
    \centering
    \includegraphics[width=0.7\textwidth]{figures/wingbox_section.png}
    \caption{Section view of the wing box highlighting internal load-carrying structure}
    \label{fig:wingbox_section}
\end{figure}

%------------------------------------------------

\subsection{Load Path Discussion}

Under positive maneuvering loads, aerodynamic lift induces an upward bending moment along the wing span. This bending places the upper skin of the wing box in compression and the lower skin in tension. The spar webs primarily carry shear loads and maintain the geometric separation between the skins.

The closed-section nature of the wing box also enables efficient resistance to torsional loads. Torque generated by aerodynamic moments is distributed around the perimeter of the section, reducing localized stress concentrations and improving overall structural stability.

%------------------------------------------------

\subsection{Structural Efficiency Considerations}

The selected wing box configuration prioritizes structural efficiency by concentrating material in regions that contribute most effectively to load carrying. By maximizing skin separation and maintaining a hollow interior, the structure achieves high bending stiffness with minimal material usage.

This approach reflects standard aerospace structural design philosophy, where weight efficiency is achieved through geometry and load-path optimization rather than excessive material thickness.

%------------------------------------------------

\subsection{Design Limitations}

The wing box design presented in this section is intended for conceptual demonstration purposes. Detailed stress sizing, buckling analysis, and fatigue assessment were not performed. Additionally, internal ribs, fasteners, and local reinforcements were not explicitly modeled.

These aspects represent important areas for future refinement but are beyond the scope of the present conceptual design study.