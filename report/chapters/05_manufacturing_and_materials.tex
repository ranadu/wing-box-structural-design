\section{Manufacturing and Material Considerations}

\subsection{Material Selection Rationale}

For the purposes of this conceptual structural design study, the wing box is assumed to be constructed from a conventional aerospace-grade aluminum alloy. Aluminum alloys are widely used in aircraft primary structures due to their favorable strength-to-weight ratio, predictable elastic behavior, corrosion resistance, and well-established manufacturing processes.

While advanced composite materials are increasingly used in modern aircraft, aluminum remains an appropriate and widely accepted baseline material for conceptual design studies. The use of aluminum also simplifies manufacturing assumptions and facilitates clearer structural load-path reasoning.

Material properties are assumed to be linear elastic, and material selection is treated at a high level consistent with early-stage design practice.

%------------------------------------------------

\subsection{Manufacturing Approach}

The wing box geometry developed in this study is compatible with conventional aerospace manufacturing methods. The structure may be fabricated using a combination of sheet metal skins and machined or formed spar webs.

The upper and lower skins would typically be manufactured as formed aluminum panels, while the spar webs could be produced through machining or extrusion, depending on design requirements. Assembly would be achieved using mechanical fasteners or bonded joints, consistent with standard aircraft production practices.

The simplified, prismatic geometry of the wing box facilitates ease of fabrication, inspection, and structural integration.

%------------------------------------------------

\subsection{Structural Assembly Considerations}

The wing box configuration allows for a modular assembly process in which individual structural elements are manufactured separately and assembled into a closed-section structure. This approach supports:

\begin{itemize}
    \item Efficient load transfer between skins and spars
    \item Ease of inspection and maintenance
    \item Flexibility for future structural modifications
\end{itemize}

Although ribs and attachment fittings are not explicitly modeled in this study, the wing box geometry readily accommodates their inclusion in later design stages.

%------------------------------------------------

\subsection{Design for Manufacturability}

From a manufacturability perspective, the wing box design prioritizes simplicity and repeatability. Straight spar webs, uniform skin thickness, and a constant cross-section reduce manufacturing complexity and minimize production risk.

These characteristics are particularly advantageous during early development phases, where design flexibility and rapid iteration are valued. The geometry can be readily adapted to include taper, sweep, or local reinforcements as the design matures.

%------------------------------------------------

\subsection{Material and Manufacturing Limitations}

The material and manufacturing considerations presented in this section are intentionally high-level. Detailed material selection, fastener design, tolerance analysis, and production cost modeling were not performed.

These elements represent critical aspects of later-stage structural development and would be addressed during detailed design and certification phases.