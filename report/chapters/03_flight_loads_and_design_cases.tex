\section{Flight Loads and Design Cases}

\subsection{Purpose of Load Definition}

Structural design decisions must be informed by representative loading conditions. For an aircraft wing, the dominant loads arise from aerodynamic lift, which generates bending moments, shear forces, and torsional effects along the span. The objective of this section is to define a set of conceptual load cases that establish the structural demands placed on the wing box.

Rather than performing a detailed loads analysis, this study identifies representative maneuver and flight conditions that are commonly used to guide early-stage wing structural design. These load cases are subsequently used to justify the wing box geometry and internal load paths presented in Section 4.

%------------------------------------------------

\subsection{Load Factor Concept}

The load factor, denoted by $n$, is defined as the ratio of aerodynamic lift to aircraft weight:

\begin{equation}
n = \frac{L}{W}
\end{equation}

where $L$ represents the lift force generated by the wing and $W$ is the aircraft weight. A load factor greater than unity indicates that the wing is supporting loads greater than those experienced during steady level flight.

Load factor provides a convenient means of expressing maneuver-induced structural demand and is widely used in conceptual and preliminary aircraft structural design.

%------------------------------------------------

\subsection{Representative Design Load Factors}

To establish credible structural design conditions, representative limit load factors were assumed based on typical values for conventional subsonic aircraft. The following values were selected for conceptual design purposes:

\begin{itemize}
    \item Positive limit load factor: $n_{+} = +3.8$
    \item Negative limit load factor: $n_{-} = -1.5$
\end{itemize}

These values are consistent with commonly referenced certification standards for normal category aircraft and provide a reasonable basis for preliminary wing structural design.

%------------------------------------------------

\subsection{V--n Diagram Interpretation}

The velocity--load factor (V--n) diagram provides a graphical representation of the aircraft operating envelope by relating allowable load factor to airspeed. At low airspeeds, the achievable load factor is limited by aerodynamic stall, while at higher airspeeds the structure becomes the limiting factor.

For the purposes of this study, the V--n diagram is used conceptually to identify critical structural design cases rather than to define exact operating boundaries. The intersection of maximum load factor and high airspeed represents the most demanding bending condition for the wing structure.

%------------------------------------------------

\subsection{Critical Structural Design Cases}

Based on the conceptual V--n envelope, the following load cases were identified as structurally significant:

\begin{itemize}
    \item Positive maneuvering load at high airspeed ($+3.8g$)
    \item Negative maneuvering load resulting in wing down-bending ($-1.5g$)
    \item Symmetric lift-induced bending during steady level flight
\end{itemize}

Among these cases, the positive maneuvering load produces the largest bending moment at the wing root and is therefore the primary driver for wing box design.

%------------------------------------------------

\subsection{Implications for Wing Box Design}

Under positive load factor conditions, the wing experiences upward aerodynamic lift, resulting in a bending moment that places the upper wing surface in compression and the lower surface in tension. The wing box must therefore provide sufficient separation between the upper and lower skins to efficiently resist bending stresses.

Shear loads generated by lift are transferred between the skins through the spar webs, while the closed-section geometry provides torsional rigidity and resistance to twist. These structural requirements directly inform the wing box geometry and internal structure developed in the following section.

%------------------------------------------------

\subsection{Limitations of Load Definition}

The load cases presented in this section are intended for conceptual design guidance only. Gust loading, asymmetric lift, and dynamic effects are not explicitly considered. A complete structural certification effort would require a comprehensive loads analysis in accordance with applicable regulatory standards.

Despite these limitations, the selected load cases provide a realistic and defensible basis for the preliminary wing box design presented in this study.